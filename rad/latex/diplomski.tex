\documentclass[times, utf8, diplomski, numeric]{fer}
\usepackage{booktabs}
\usepackage{url}
\usepackage{hyperref}
\usepackage[parfill]{parskip}

\begin{document}

% TODO: Navedite broj rada.
\thesisnumber{000}

% TODO: Navedite naslov rada.
\title{Proširivi sustav za stvaranje CTF zadataka specifičnih za sustave upravljanja i
zabave u automobilima}

% TODO: Navedite vaše ime i prezime.
\author{Lovro Grgurić Mileusnić}

\maketitle

% Ispis stranice s napomenom o umetanju izvornika rada. Uklonite naredbu \izvornik ako želite izbaciti tu stranicu.
\izvornik

% Dodavanje zahvale ili prazne stranice. Ako ne želite dodati zahvalu, naredbu ostavite radi prazne stranice.
\zahvala{}

\tableofcontents

\chapter{Uvod}
Današnji automobili i ostala cestovna vozila značajno se razlikuju od njihovih prvih inačica, iako im glavna namjena ostala ista. Pojavom i širom primjenom elektronike, automobili više nisu samo mehanički strojevi, već sadrže određen broj međusobno povezanih elektroničkih upravljačkih jedinica (engl. \textit{electronic control unit}, ECU) \cite{koscher2010}. Elektroničke upravljačke jedinice svojom međusobnom suradnjom omogućavaju sigurno upravljanje vozilom, ali i dodatne funkcionalnosti poput klimatizacije te sustava za informacije i zabavu (engl. \textit{in-vehicle infotainment}). Većina komunikacije između elektroničkih upravljačkih jedinica odvija se putem protokola poput \textit{Controller Area Network} (CAN) protokola i pripadajućih sabirnica. Prvotno namijenjen za primjenu u automobilima, CAN ima svojstva prikladna za komunikaciju u stvarnom vremenu (engl. \textit{real time}), kao i sposobnosti detektiranja grešaka i propuštanja poruka višeg prioriteta\cite{canopen1}. Međutim, iako ga specifikacija druge inačice CAN protokola, CAN 2.0, opisuje kao protokol s visokom razinom sigurnosti, ni ta niti originalna specifikacija ne razmatraju osiguravanje osnovnih sigurnosnih svojstava povjerljivosti, integriteta i dostupnosti (engl. \textit{confidentiality, integrity, availability}, CIA) \cite{bosch1991, canislabs1}. Zbog drastične prirode posljedica koje bi neispravan rad sustava automobila mogao imati na vozača i njegove suputnike te ostale sudionike prometa, pri inženjeringu sustava automobila uvijek je bila pridodana posebna pažnja njihovoj funkcionalnoj sigurnosti (engl. \textit{safety}) i ispravnosti, ali ne nužno i kibernetičkoj sigurnosti\cite{koscher2010}. Zbog tadašnje zatvorenosti komunikacije ovih sustava, za potencijalnog napadača nije postojala površina (engl. attack surface) koju bi mogao iskoristiti bez fizičkog pristupa CAN sabirnici.

Interne mreže današnjih vozila kao i programska podrška njihovih sustava znatno su kompleksnije i povezanije s vanjskim svijetom\cite{huq2020driving}. Mnogi novi modeli automobila su opremljeni SIM karticama odnosno mogućnošću povezivanja na mobilnu mrežu u svrhu komunikacije sa servisima proizvođača u oblaku (engl. cloud services) i mobilnim aplikacijama koje omogućuju udaljeno upravljanje vozilom. Uz to, vozila povezana na mobilnu mrežu koriste ju i u svrhu preuzimanja ažuriranja (engl. \textit{Over-The-Air updates}, OTA) za sustave zabave, ali i za kritične upravljačke sustave i elektroničke upravljačke jedinice. \textit{Bluetooth} tehnologija najčešće se koristi za povezivanje mobilnog telefona sa sustavom zabave kako bi se omogućilo pozivanje i reprodukcija glazbe. U svrhu bolje integracije mobilnih telefona u sustav zabave, mnogi proizvođači podržavaju platforme poput \textit{Android Auto} ili \textit{Apple Carplay}, kojima se sustav zabave proširuje mogućnostima preuzimanja i pokretanja raznovrsnih aplikacija iz izvora treće strane. Vozila opremljena mrežnim karticama s mogućnošću stvaranja i korištenja Wi-Fi bežičnih pristupnih točaka, mogu dijeliti pristup mobilnoj mreži koji ostvaruju s prethodno navedenim SIM karticama ili povezivati se na pristupne točke korisnika u svrhu preuzimanja OTA ažuriranja ili aplikacija. Za skoro svaku od navedenih funkcionalnosti istraživači automobilske sigurnosti pokazali su mogućnost njihvog iskorištavanja i ulančavanja otkrivenih ranjivosti u svrhu neautoriziranog pristupa i postizanja kontrole nad sustavima zabave, ali i kritičnim sustavima upravljanja automobila \cite{nie2017free, nie2018over, cai20190, tencent2018bmw, miller2015remote, curry2023web}. Nadalje, već u izvješću o stanju automobilske sigurnosti tvrtke \textit{Upstream Security} iz 2019. godine, predviđa se da će sva nova vozila proizvedena u 2025. godini imati navedene funkcionalnosti, ali i da će podržavati komunikaciju među vozilima (engl. \textit{Vehicle-to-Vehicle, V2V}) i komunikaciju vozila s infrastrukturom (engl. \textit{Vehicle-to-Infrastructure, V2I}), dodatno povećavajući dostupnu površinu napada \cite{upstream2019report}.

Nagli rast povezanosti i kompleksnosti sustava vozila stvara potrebu za obrazovanjem određenog broja stručnjaka i inženjera koji će testirati i osiguravati kvalitetniju kibernetičku sigurnost tih sustava, od razine sklopovlja do \textit{backend} servisa te V2V i V2I komunikacije. Osim formalnih sustava obrazovanja, popularan format za edukaciju stručnjaka kibernetičke sigurnosti su i \textit{Capture the Flag} (CTF) zadaci. Međutim, u \cite{vsvabensky2021cybersecurity} autori ističu da se u analiziranom uzorku od 12 952 \textit{Capture the Flag} zadataka, zadaci klasificirani u područje sigurnosti komponenti (engl. \textit{component security}) čine tek 8,19\%, što je 6. najmanji udio od ukupno 8 analiziranih područja znanja. Većina današnjih \textit{Capture the Flag} zadataka spada u razvijenije i poznatije grane kibernetičke sigurnosti poput \textit{web} sigurnosti, sigurnosti operacijskih sustava i aplikacija te sigurnosti oblaka (engl. \textit{cloud security})\cite{prinetto2020hardware}. U svrhu edukacije sigurnosnih stručnjaka u području kibernetičke sigurnosti u automobilskoj industriji potrebno je napraviti proširivi sustav koji će služiti kao temelj za stvaranje \textit{Capture the Flag} zadataka koji sadrže specifičnosti sustava u automobilima.

Rad je strukturiran kroz 6 poglavlja. U prvom poglavlju nakon uvoda, odnosno drugom poglavlju cjelokupnog rada, napravljen je pregled komunikacijskih i dijagnostičkih protokola te arhitekture modernog automobila. Kroz ovo poglavlje bit će opisani protokoli korišteni u današnjim automobilima kao i sustavi koje povezuju, čime se postavlja teoretska i tehnička podloga za razumijevanje napada na iste, opisanih u trećem poglavlju. U četvrtom poglavlju razmatraju se postojeći sustavi za obuku stručnjaka sigurnosti u grani sigurnosti automobila, ali i u ostalim granama sigurnosti te je definiran opis svojstava takvog sustava. U petom poglavlju opisana je implementacija proširivog sustava za stvaranje \textit{Capture the Flag} zadataka specifičnih za sustave upravljanja i zabave u automobilima, kao i tehničke odluke donesene tijekom njegove razrade. Naposljetku, u zaključku je opisan je konačni rezultat, prednosti i mane implementiranog sustava te njegova moguća unaprjeđenja i smjernice za daljnji razvoj.

\chapter{Sustavi i protokoli modernih automobila}
Kao temelj za razumijevanje iskoristivih površina i mogućih vektora napada na sustave automobila, kroz ovo poglavlje opisana je električna i elektronička arhitektura modernog automobila (engl. \textit{Electrical/Electronic} architecture). Uz to, opisani su i komunikacijski te dijagnostički protokoli specifični za automobile.
\section{E/E arhitektura modernog automobila}
E/E arhitektura obuhvaća elektroničke komponente, električko ožičenje, tehnologije umrežvanja i programsku podršku jednog automobila \cite{nasser2023automotive}. Prema \cite{nasser2023automotive, koscher2010, knight2020hacking, huq2020driving, aliwa2021cyberattacks}, sustavi vozila dijele se prema funkciji u domene:

\begin{itemize}
    \item pogonskog sklopa (engl. \textit{powertrain})
    \item šasije (engl. \textit{chassis})
    \item kabine (engl. \textit{interior cabin} ili \textit{body})
    \bigskip
    \item zabave i povezivosti (engl. \textit{infotainment and connectivity})
\end{itemize}

Iako se sustavi mogu prema funkciji podijeliti u navedene 4 domene, domene nužno ne određuju i konfiguraciju mreže automobila.

Domene pogonskog sklopa, šasije i kabine sadrže ECU-ove koji najviše utječu na fizičko stanje automobila te je njihova komunikacija većinom ograničena na mrežu automobila. Domena informacija, zabave i povezivosti sadrži sustave koji pružaju dodatne informacije i sadržaje vozaču i putnicima te vrši komunikaciju s \textit{backend} servisima, mobilnim aplikacijama i mobilnim uređajima. Stoga je domena informacija, zabave i povezivosti obrađena u zasebnom potpoglavlju.   
\subsection{Domene pogonskog sklopa, šasije i kabine}
Elektroničke upravljačke jedinice odnosno ECU-ovi, pripadaju domenama pogonskog sklopa, šasije i unutarnje kabine te služe za koordiniranje i upravljanje većinom funkcija automobila. ECU-ovi detektiraju trenutne operativne uvjete pomoću senzora, procesuiraju ih i aktiviraju aktuatore\cite{bosch2022handbook}. Sklopovlje ECU-a najčešće se nalazi u zatvorenom zaštitnom kućištu s izloženim priključcima za spajanje na ostatak mreže automobila, senzore i aktuatore. Sklopovlje je najčešće tiskana pločica na kojoj se najčešće nalazi integrirani krug za upravljanje snagom (engl. \textit{power management integrated circuit}, PMIC), primopredajnici protokola kojim komuniciraju, memorije te mikrokontroler ili sustav na čipu (engl. \textit{System on a chip}), SoC)\cite{nasser2023automotive}. ECU-ovi ovih domena najčešće zbog \textit{real time} zahtjeva međusobno komuniciraju putem CAN ili FlexRay protokola, gdje specifična topologija mreže ovisi o modelu automobila \cite{bosch2022handbook}.

Domeni pogonskog sklopa pripadaju ECU-ovi koji upravljaju i utječu na funkciju motora i prijenosa, a to su upravljački modul motora (engl. \textit{engine control module}, ECM) i upravljački modul prijenosa (engl. \textit{transmission control module}, TCM). U literaturi se često koriste i nazivi upravljačka jedinica motora (engl. \textit{engine control unit}, ECU), odnosno upravljačka jedinica prijenosa (engl. \textit{transmission control unit}, TCU)\cite{nasser2023automotive, koscher2010}. Često automobili imaju jedan snažniji ECU koji obavlja obje funkcije, upravljanje motorom i upravljanje prijenosom te se takav ECU naziva upravljačkom jedinicom pogonskog sklopa (engl. \textit{powertrain control unit}, PCU) ili upravljačkim modulom pogonskog sklopa (engl. \textit{powertrain control module}, PCM) \cite{bosch2022handbook, ecutesting}. U slučaju vozila na električni pogon, u domeni pogonskog sklopa pripadaju sustav upravljanja baterijom (engl. \textit{battery managment system}, BMS), inverter \engl{inverter} i PCU. Uz navedene, u domenu pogonskog sklopa ulaze i druge upravljačke jedinice koje prikupljaju podatke sa senzora i upravljaju aktuatorima relevantnima pogonskom sklopu. S obzirom na to da navedeni ECU-ovi imaju izravan utjecaj na kretanje automobila, njihovo kompromitiranje može rezultirati fizičkim ozljedama vozača i putnika.

U domenu šasije svrstavaju se sustavi i senzori odgovorni za funkcije zaštite putnika koje zahtijevaju reakciju u stvarnom vremenu poput sustava kočenja, ovjesa i zračnih jastuka. Sukladno tome, kao i u slučaju sustava domene pogonskog sklopa, njihovo ispravno funkcioniranje je iznimno bitno za sigurnost vozača i putnika\cite{nasser2023automotive}. Jedna od upravljačkih jedinica koje pripadaju u domenu šasije je upravljački modul elektroničkog kočenja (engl. \textit{electronic braking control module}, EBCM). EBCM je specijalizirani modul koji upravlja aktivnim sigurnosnim mjerama poput ABS-a (njem. \textit{Antiblockiersystem}), elektroničkom kontrolom stabilnosti (engl. \textit{electronic stability control}, ESC) te automatskog hitnog kočenja (engl. \textit{automated emergency breaking}, AEB). Uz EBCM, u domenu šasije svrstavaju se i napredni sustav za podršku vozaču pri upravljanju vozilom (engl. \textit{Advanced driver-assistance system}, ADAS), koji je sve češće prisutan u novim vozilima\cite{nasser2023automotive}. ADAS je odgovoran za funkcije poput rada tempomata, održavanja razmaka između vozila,  automatskog zadržavanja trenutne prometne trake, automatskog parkiranja te autonomne vožnje \cite{bosch2022handbook}. Domeni šasije pripada i upravljački modul zračnih jastuka (engl. \textit{airbag control module}, ACM) te sustav servo upravljača (engl. \textit{electronic power steering, EPS}. ADAS i EBCM prikupljaju podatke od niza sustava senzora, primjerice od radara za određivanje udaljenost objekata u svrhu aktivacije funkcije AEB, od ultrazvučnih senzora za potrebe sustava automatskog parkiranja te od sustava lidar i video senzora za potrebe autonomne vožnje. 

U domenu kabine spadaju sustavi manje kritičnih funkcija te uključuju sustave grijanja, ventilacije i klimatizacije (engl. \textit{heating, ventilation and air conditioning}, HVAC), pripadajući upravljački modul klimatizacije (engl. \textit{climate control module}, CCM) te ECU-ove za prikupljanje podataka sa senzora sigurnosnih pojasa i sjedala te ECU-ove za upravljanje osvjetljenjem, prozorima, brisačima i ostalim dijelovima kabine. Uz navedene, kao sigurnosno bitni sustavi, ističu se sustav za ulaz bez ključa (engl. \textit{keyless entry system}, KES) te ECU za upravljanje otključavanjem vrata vozila i pripadajući prijemnik za daljinsko zaključavanje (engl. \textit{remote control door lock receiver}, RCDLR). KES i RCDL sustavi najčešće su iskorištavani sustavi u slučaju krađe vozila, često putem \textit{relay} napada \cite{nasser2023automotive, cbc2020relay}.

\subsection{Domena informacija, zabave i povezivosti}
Suprotno domenama pogona, šasije i kabine, domena informacija, zabave i povezivosti sadrži sustave čije funkcionalnosti većinom nisu visokog prioriteta niti su uvjetovane \textit{real time} zahtjevima. Sustav za informacije i zabavu, odnosno IVI, obuhvaća niz podsustava koji vozaču i putnicima pružaju informacije o stanju vozila i dodatne sadržaje. 

Ploča s instrumentima \engl{instrument cluster}, u mehaničkoj izvedbi ili kao digitalni zaslon, pruža vozaču informacije o brzini vozila, broju obrtaja motora i količini preostalog goriva te stanju pokazivača smjera, tempomata i kvarova kroz određen broj indikatorskih svjetlećih dioda. U digitalnoj izvedbi, ploča s instrumentima može prikazivati i druge korisne informacije, kako vozač ne bi morao skretati pogled na središnji zaslon tijekom vožnje, poput trenutne radio postaje, parking kamere te navigacije \cite{bosch2022handbook, nasser2023automotive}. 

Navigacija se često prikazuje i na središnjem zaslonu u sklopu grafičkog sučelja operacijskog sustava središnje jedinice \engl{head unit}. Središnji zaslon je često osjetljiv na dodir ili je njime moguće upravljati pomoću fizičkih tipki na središnjoj konzoli. Putem grafičkog sučelja vozač i putnici vozila mogu iskoristiti mogućnost povezivanja mobilnog telefona sa sustavom za informacije i zabavu putem \textit{Bluetooth} tehnologije ili USB-a, u svrhu reprodukcije glazbe i telefoniranja.

Noviji modeli automobila podržavaju integraciju s platformama poput \textit{Android Auto} i \textit{Apple Carplay}, koji omogućavaju puno dublju integraciju funkcija mobilnog telefona u sustave informacija i zabave. Povezivanjem telefona s njemu pripadajućom platformom, korisnicima automobila omogućava korištenje aplikacija za navigaciju, reprodukciju glazbe i ostalih multimedija, korištenje \textit{web} preglednika, telefoniranje i razmjenu poruka s njihovog mobilnog uređaja. Omogućavaju i integraciju obavijesti, kalendara te "pametnih" asistenata \cite{androidauto, carplay}.

Primarna funkcija upravljačke jedinice za telematiku (engl. \textit{telematics control unit}, TCU), je omogućavanje povezivosti automobila putem mobilne mreže, \textit{Wi-Fi} i \textit{Bluetooth} tehnologija te prijam GPS signala\cite{nasser2023automotive}. Povezivost putem \textit{Wi-Fi} tehnologije i mobilne mreže koristi se u svrhu transmisije telemetrijskih podataka, preuzimanja OTA ažuriranja te komunikaciju s \textit{backend} servisima proizvođača, ali i u svrhu zaštite putnika pozivanjem hitnih službi u slučaju prometne nesreće.

Operacijski sustav središnje jedinice najčešće je utemeljen na \textit{Linux}, \textit{QNX} i \textit{Android} operacijskim sustavima ili operacijskom sustavu proizvođača (engl. \textit{proprietary} \cite{nasser2023automotive}. Operacijski sustav objedinjuje sve navedene funkcionalnosti te omogućava upravljanje istima putem središnjeg zaslona, konzole i te putem glasa \cite{bosch2022handbook}. Zbog visoke razine kompleksnosti i povezivosti, središnja jedinica ima najveću površinu napada te je česta probojna točka u ostatak sustava automobila u mnogim istraživanjima sigurnosti automobila \cite{aliwa2021cyberattacks, knight2020hacking, smith2016car, tencent2018bmw, miller2015remote}.

\subsection{Ostale komponente}
Od ostalih komponenti E/E mreže, bitno je spomenuti OBD-II priključak za dijagnostiku. OBD-II priključak omogućava serviserima brzo očitavanje dijagnostičkih kodova neispravnosti (engl. \textit{diagnostic trouble code}, DTC) te se u osobnim automobilima najčešće nalazi ispod upravljača automobila \cite{smith2016car}. Uz to, OBD-II priključak omogućava pristup CAN-u automobila te ukoliko je mreža neispravno segregirana moguće je putem OBD-II priključka  komunicirati s ECU-ovima kritičnima za sigurnost putnika \cite{knight2020hacking, smith2016car}. Potencijalni napadač koji već ima pristup unutrašnjosti automobila može iskoristiti neispravnu segregaciju mreže. Primjerice, spajanjem zloćudnog uređaja na OBD-II priključak može onemogućiti upravljački modul elektroničkog kočenja tijekom vožnje, ugrožavajući fizičku sigurnost putika.  

\section{Tipovi E/E arhitektura}
Tip E/E arhitekture može utjecati na površinu napada vozila te omogućiti ili onemogućiti određene vektore napada. U slučaju potpuno nesegregiranih E/E arhitektura, što je karakteristično za starije modele automobila, kompromitiranjem jednog od sustava automobila moguće je komunicirati sa svim ostalim sustavima. Jedan od modela s nesegregiranom E/E arhitekturom je \textit{Jeep Cherokeee} iz 2014. godine kojeg su uspjeli kompromitirati Miller i Valasek, nakon čega su proizvođači segregaciju E/E arhitektura krenuli provoditi i na jeftinijim modelima \cite{miller2015remote, dissecto2023networks}

U \cite{bosch2022handbook} autori razlikuju tri tipa E/E arhitektura, navedene povijesnim redoslijedom:
\begin{itemize}
    \item Funkcionalno raspodijeljena E/E arhitektura \engl{Functionally distributed E/E architecture}
    \item E/E arhitektura centralizirana po domenama \engl{Domain-centralized E/E architecture}
    \item Zonalna E/E arhitektura (engl. \textit{Zone E/E architecture} 
\end{itemize}

U današnjim osobnim automobilima je još uvijek je najčešća funkcionalno raspodijeljena arhitektura najčešća, dok se E/E arhitektura centralizirana po domenama sve češće pojavljuje u novijim modelima vozila. Zonalna E/E arhitektura je još uvijek u razvoju, ali se smatra da će zbog svoje skalabilnosti postati dominantna\cite{bosch2022handbook, nasser2023automotive, dissecto2023networks}.

\subsection{Funkcionalno raspodijeljena E/E arhitekture}
U funkcionalno raspodijeljenim E/E arhitekturama, grupirani su ECU-ovi sličnih funkcionalnosti te su međusobno povezani CAN ili FlexRay sabirnicama. Pojedinačne sabirnice mogu biti povezane središnjim poveznikom \engl{central gateway}. Središnji poveznik vrši funkciju koordiniranja ponašanja između ECU-ova različitih sabirnica te segregiranja ECU-ova kritičnih za sigurnost putnika od ECU-ova manje kritičnih funkcionalnosti, a nekada i veće površine napada, primjerice ECU-ova iz domene informacija, zabave i povezivosti.

\subsection{E/E arhitektura centralizirana po domenama}
E/E arhitektura centralizirana po domenama uvodi domenske upravljačke jedinice (engl. \textit{domain control unit}), odnosno ECU-ove s više procesorske snage, koje obnašaju funkciju više manjih ECU-ova iz iste domene. Opcionalno, DCU-ovi različitih domena mogu biti povezani središnjim poveznikom ili izravno, najčešće putem \textit{Etherneta} \cite{bosch2022handbook, nasser2023automotive}.

\subsection{Zonalna E/E arhitektura}
Zonalna E/E arhitektura centralizira logiku u središnjem računalu vozila \engl{central vehicle computer} te uvodi zonske upravljače te ih dijeli prema fizičkoj poziciji unutar vozila. Prednost zonalne arhitekture je dobra skalabilnost pri povećanju broja senzora \cite{bosch2022handbook}. S obzirom na to da se senzori mogu nalaziti u bilo kojem dijelu vozila, izravno ih povezivati na ECU-ove kojima pružaju podatke nije jednostavno izvedivo. Zonski upravljači služe kao agregatori podataka sa senzora te ih proslijeđuju središnjem računalu vozila koje ih procesuira ili proslijeđuje u druge zone. Središnje računalo vozila sastoji se od više centralnih i grafičkih procesorskih jedinica te jezgara za zadatke s \textit{real time} vremenskim zahtjevima, kako bi podržale računalno intenzivne zadatke poput autonomne vožnje\cite{nasser2023automotive}.
\section{Poopćena E/E arhitektura modernog automobila}
\section{Komunikacijski protokoli}
\subsection{CAN}
\subsection{LIN}
\subsection{FlexRay}
\subsection{MOST}
\subsection{Automotive Ethernet}
\section{Dijagnostički protokoli}
\subsection{OBD-II}
\subsection{UDS}
\subsection{XCP}

\chapter{Pregled postojećih napada}
\section{Napadi na komunikacijske i dijagnostičke protokole}
opcenito opisi za ove koje ne razmatramo u detalje
\subsection{CAN}
\subsection{UDS}
\subsection{XCP}
\section{Napadi na sustave automobila}
\chapter{Postojeći edukacijski materijali u području sigurnosti vozila}
hw based caputre the flag challenges,
blockharbour ctf
htb
onaj guide od valseka
... pogledaj biljeske
\chapter{Implementacija sustava za obuku stručnjaka sigurnosti}

spomeni xcp scapy ogranicenj
\chapter{Zaključak}
 Zaključak.

\bibliography{literatura}
\bibliographystyle{fer}


\begin{sazetak}
Sažetak na hrvatskom jeziku.

\kljucnerijeci{Ključne riječi, odvojene zarezima.}
\end{sazetak}

% TODO: Navedite naslov na engleskom jeziku.
\engtitle{Title}
\begin{abstract}
Abstract.

\keywords{Keywords.}
\end{abstract}

\end{document}
